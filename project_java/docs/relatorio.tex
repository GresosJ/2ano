\documentclass[11pt]{article}

% Language setting
\usepackage[portuguese]{babel}

% Set page size and margins
\usepackage[a4paper,top=2cm,bottom=2cm,left=3cm,right=3cm,marginparwidth=1.75cm]{geometry}

% Useful packages
\usepackage{amsmath}
\usepackage{graphicx}
\usepackage{minted}
\usepackage{fancyhdr}
\usepackage{lastpage}
\usepackage[all]{xy}
\usepackage[colorlinks=true, allcolors=blue]{hyperref}

\title{Laboratórios de Informática III}
\author{Grupo 43 \\ 62608 - Marco Sousa \\ 93271 - José Malheiro \\ 88000 - Gerson Junior} 

\begin{document}
\maketitle

\begin{abstract}
	Este projeto permitiu desenvolver competências de \textbf{Engenharia de Software}, nomeadamente \textit{modularidade, encapsulamento, reutilização e escalabilidade} de programas.
	
	Foi utilizada a arquitetura \textit{Model - View - Controller} com estratégia \textit{facade}, juntamente com a linguagem de programação \textit{Java}.
	
	Num contexto semelhante ao trabalho anteriormente elaborado, programou-se um \textbf{Sistema de Gestão de Reviews}, incluindo uma interação I/O dinâmica, carregamento e gravação de informação, bem como uma gestão interna desta, otimizada através da comunicação efetiva elaborada entre as diferentes classes criadas.
\end{abstract}

\section{Introdução}

O presente relatório foi redigido no âmbito da unidade curricular (UC) \underline{Laboratórios de Informática} e remete-se à elaboração de um projeto na liguagem de programação \textit{Java} para um \textbf{Sistema de Gestão de Reviews}.

A construção do projeto teve como referência a orientação dos docentes da UC e principal objetivo de desenvolver
conceitos de modularidade, encapsulamento, construção de código reutilizável e otimização através da escolha de estratégias para aumentar a rapidez do programa.

Em adição, permitiu o aprofundamento sobre o desenvolvimento de programas na linguagem Java.

\section{Estrutura do Projeto}

Optou-se por um modelo que baseia-se nos princípios do MVC \textit{(Model View Controller)}, mas em conjunto com uma estratégia \textit{facade}, 
para diminuir as dependências entre classes, encapsular e esconder aquelas que se encontram atrás, de modo a evoluirem de forma autónoma;
tudo no sentido de uma melhor organização e modularização do código.

Tomou-se, ainda, a liberdade de construir interfaces para basicamente todas as classes que constituem o projeto,
de modo a aumentar a \textit{abstração} e tornar o código mais flexível;
o programa fica menos volátil e frágil, sendo possível variações na sua implementação.  

A arquitetura criada compreende, essencialmente, seis camadas de desenvolvimento:

\begin{itemize}
	\item Domain
		\begin{itemize}
			\item Estruturas de dados básicas
			\item Lógica inicial 
		\end{itemize}
	\item Services
		\begin{itemize}
			\item Catálogos das classes base
			\item Expande a lógica inicial para um maior grupo de objetos
			\item Permitem estabelecer a relação entre as estruturas
			\item Facilita a manipulação dos dados
		\end{itemize}
	\item GestReviews
		\begin{itemize}
			\item Classe agregadora 
			\item Utiliza a lógica do \textit{Domain} extendida por \textit{Services}
			\item Calcula a informação a ser passada ao \textit{Controller} 
		\end{itemize}
	\item Middleware
		\begin{itemize}
			\item Exceções criadas
			\item Melhora a visualização de possíveis erros
			\item Providência um \textit{debug} facilitado
		\end{itemize}
	\item Controller
		\begin{itemize}
			\item Ponto de entrada do utilizador
			\item Comunica com o \textit{GestReviews} para gerar dados
		\end{itemize}
	\item View	
		\begin{itemize}
			\item É atualizado pelo \textit{Controller}
			\item Apresenta a informação ao utilizador
			\item É o ponto de saída da informação
		\end{itemize}
\end{itemize}

A interação com o utilizador foi conseguida através do desenvolvimento de várias \textit{frames} que facilitam a visualização do trabalho, a escolha das consultas iterativas e a inserção dos valores no programa. 

A execução do programa tem três propósitos principais:

\begin{enumerate}
	\item Ler ficheiros e carregar/popular a estrutura interna
	\item Apresentar valores estatísticos dos ficheiros carregados
	\item Atribuir um resultado para cada consulta e visualizá-lo
\end{enumerate}

\begin{equation*}
	\xymatrixcolsep{4pc}\xymatrix {
		& & & CSV or .dat\ar@{.o}[dl]_{load} \\
	View\ar@<1ex>[r]\ar@<1ex>[d] & Controller\ar@<1ex>[r]\ar@<1ex>[l] & GestReviews\ar@<1ex>[l]\ar@{.o}[r]_{save} & .dat \\
	User\ar@<1ex>[u] \\
	& IBusRepo\ar[uur] & IUserRepo\ar[uu] & IRevRepo\ar[uul] & IStatsRepo\ar[uull] \\
	& IBusiness\ar[u] & IUser\ar[u] & IReview\ar[u]
	}
\end{equation*}

\subsection{Estruturas de Dados}

As estruturas de dados principais criadas para representar a informação necessária para a gestão de Reviews foram \underline{Review}, \underline{Business} e \underline{User}.

No sentido de garantir o encapsulamento da informação, as classes criadas foram desenvolvidas de modo a que estrutura seja desconhecida por quem utiliza a API.

Assim, são disponibilizados métodos que constroem, acedem e alteram a classe e as suas variáveis de instância (por exemplo, Construtores, \textit{getters} e \textit{setters}).

É implementado, em adição a estes métodos, a interface \textit{Comparable} para ser possível a comparação entre instâncias do mesmo objeto.

A relação estabelecida entre as estruturas principais pode ser descrita como:

\begin{equation*}
	\xymatrixcolsep{5pc}\xymatrix {
	& Review \\ Business\ar[ur] |-{One-to-Many}\ar[rr] && User\ar[ul] |-{One-to-Many}\ar[ll]^{Many-to-Many}
	}
\end{equation*}

\vspace{1pc}
Na package do \textit{Domain} também podemos contar com classes que vão entrar no novo parâmetro de estatísticas adicionado, como o caso das estruturas \underline{FileRead}, auxiliar de \underline{FilesRead} e \underline{Crono}.

Adicionado a estes temos estruturas, como \textit{Accumulator} e \textit{KeySetValue}, criadas por utilidade, no âmbito de melhorar a forma de guardar a informação nas outras classes.

Para todas as classes inferiores, para além de serem criadas \textit{interfaces} próprias, foi ainda implementada a interface \textit{Serializable} para permitir a leitura em binário. 

Devido à hierarquia do projeto as classe superiores vão também herdar esta implementação.

\subsubsection{Review} \label{review}

\begin{minted}{java}

public class Review implements IReview, Serializable, Comparable<Review> {
	private String review_id;
	private String user_id;
	private String business_id;
	private Double stars;
	private Integer useful;
	private Integer funny;
	private Integer cool;
	private LocalDateTime date;
	private String text;
}
\end{minted}

Armazena a informação de cada Review. Ver \ref{review_api}.

Na sua interface \textbf{IReview} estão incluídos todos os \textit{getters} e \textit{setters} que são necessários para fazer alterações, bem como o método \textit{equals}, \textit{clone} para auxiliar na criação e validação de novas intãncias da classe e, por fim, um método \textit{compareTo} para comparar dois objetos.

Foi criada, para o auxílio do programa e como sugestão dos docentes da UC, uma classe dedicada a testar estruturas do tipo \textit{Review}, em \textbf{ReviewTest}, que analisa a execução da classe, nomeadamente o bom funcionamento de construtores, \textit{getters} etc.

Por fim, é notável a criação de um construtor que recebe especificamente a \textit{String} com o \textbf{Input} de uma linha do ficheiro \textit{CSV}, sendo efetuado o \textbf{parsing} dela e populada a estrutura.

\subsubsection{Business}

\begin{minted}{java}
	
public class Business implements IBusiness, Serializable, Comparable<Business> {

	private String id;
	private String name;
	private String city;
	private String state;
	private Set<String> categories;
}
\end{minted}

Estrutura criada para guardar a informação de um negócio (Business). Ver \ref{bus_api}.

A \textit{interface} associada é \textbf{IBusiness} e, de igual forma à estrutura anterior, possui métodos como os \textit{getters}, \textit{setters}, etc.

Para além da criação de métodos, como \textit{compareTo} e o construtor pelo \textit{Input}, foi também criada uma classe de Teste em \textbf{BusinessTest}.

\subsubsection{User}

\begin{minted}{java}

public class User implements IUser, Serializable, Comparable<User> {

	private String id;
	private String name;
	private Set<String> friends;
}	
\end{minted}

Armazena a informação de cada utilizador. Ver \ref{usr_api}.

Contém a sua \textit{interface} em \textbf{IBusiness} e a sua classe de teste em \textbf{UserTest} e, como nas anteriores, possuí o método comparaTo devido a implementação da interface \textit{Comparable}.

\subsubsection{Estatísticas}

\subsubsection{Estruturas Auxiliares}

\subsection{Repositórios de Informações - Catálogos}

\subsubsection{ReviewRepository}

\subsubsection{BusinessRepository}

\subsubsection{UserRepository}

\subsubsection{UserStats}

\subsubsection{BusStats}

\subsubsection{StatsRepository}

\subsection{Sistema Gestão de Recursos - GestReviews}

query moment

\subsection{Controller}

<<<<<<< Updated upstream
\subsection{Considerações Finais}
=======
\subsection{View}
>>>>>>> Stashed changes

\section{Anexos}

\subsection{Review API} \label{review_api}

\inputminted{java}{../src/Domain/Interfaces/IReview.java}

\subsection{Business API} \label{bus_api}

\inputminted{java}{../src/Domain/Interfaces/IBusiness.java}

\subsection{User API} \label{usr_api}

\inputminted{java}{../src/Domain/Interfaces/IUser.java}

\end{document}